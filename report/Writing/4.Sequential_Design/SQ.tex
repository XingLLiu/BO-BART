
\subsection{Sequential Bayesian Quadrature with BART}
\label{Sequential Design}

We propose a simple and elegant sequential design approach to BQ using BART. The idea of sequential design is to select samples to improve model performance by adding more training data to the original data set, through an active learning procedure.

Our method, summarised in Algorithm~\ref{alg:SQ} is as follows: we randomly generate a set of candidates
$\mathcal{C} = \{c_1, \ldots, c_L\}\sim p(x)$.  Using our existing BART model which we have already fit $f_{\bart}$, we use this model to calculate the posterior predictive variance for each candidate. Our selection criterion is very simple: the point that we are least certain about is the one that we will query next. Thus we maximize the posterior predictive variance:
\begin{equation}
	c^* = \mbox{argmax}_{c \in C} \mbox{Var}[f_{\bart}(c) |\mathcal{D}]
	%\times p(c)],
	\label{eq:maxvariance}
\end{equation}
%where $\widehat{\mbox{Var}}[\cdot]$ is the sample variance and $f_j$ is the sum of trees in the $j^{th}$ posterior draw. 
%{\color{red} Intuitively, such sample a $c^{*}$ would result in a maximal variance band, in which the true integral $I_f$ is more likely to be covered}. For $p(x)$ being the uniform probability density, $p(c)$ is some constant $C$ for all $c\in\mathbb{R}^d$, so assuming that the candidates in the set $\mathcal{C}$ are in $\mathcal{X}^d$, the support of $f$, we can simplify the above to:
%\begin{equation}
%	c^* = \mbox{argmax}_c \widehat{\mbox{Var}}[(f_{1}(c),..,f_{N}(c))].
%	\label{eq:Maxvar}
%\end{equation}
%This procedure combined with BART-BQ is summarised in Algorithm~\ref{alg:SQ}.

Having selected $c^*$ we query the true function $f$ to obtain $y_c^* = c^*$ and update our training set.

\begin{algorithm}[tbh!]
  \caption{Sequential Design}
  \label{alg:SQ}
\begin{algorithmic}
  \STATE {\bfseries Input:}\\training set $\mathcal{D} = \{x_1, \ldots, x_n\}$, \\response $\mathcal{Y} = \{y_1, \ldots, y_n\}$, \\
  number of iterations $M$ \\
  probability density function $p(x)$
  \FOR{$M$ iterations}
  \STATE obtain $N$ samples from the posterior distribution of the BART model with data $\mathcal{D}$ and $\mathcal{Y}$
  \STATE sample candidate set $\mathcal{C} = \{c_1, \ldots, c_L\}\sim p(x)$ 
  \STATE find $f_{\bart}^{(1)}(c), \ldots, f_{\bart}^{(N)}(c)$ for all $c\in\mathcal{C}$
  \STATE find $\,c^* = \mbox{argmax}_{c} \mbox{Var}[f_{\bart}(c)|\mathcal{D}]
	\label{eq:maxvar}$
  \STATE find response $y_c^* = f(c^*)$
  \STATE $\mathcal{D} \leftarrow \mathcal{D}\cup \{c^*\}$, $\mathcal{Y} \leftarrow \mathcal{Y}\cup \{y_c^*\}$
  \ENDFOR
\end{algorithmic}
\end{algorithm}