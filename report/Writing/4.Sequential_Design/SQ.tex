The major work of BQ with BART is to improve model performance by adding more training data to the original data set, through an active learning procedure.

The way of choosing training point is to randomly generate a set of candidates $ \mathcal{C} = \{ c_i\}_{i\in I}$ from a uniform distribution (or using a candidate set in practice, as described in Section \ref{Real Data}) and select the most informative one by calculating its posterior variance in $f(c)$ weighted by $p(c)$:
\begin{equation}
	c^* = \mbox{argmax}_c \widehat{\mbox{Var}}[(f_{1}(c),..,f_{N}(c))\times p(c)],
	\label{eq:maxvariance}
\end{equation}
where $\widehat{\mbox{Var}}[\cdot]$ is the sample variance and $f_j$ is the sum of trees in the $j^{th}$ posterior draw. For $p(x)$ being the uniform distribution, $p(c)$ is some constant $C$ on its support $\mathcal{X}$, so assuming that the candidates in the set $\mathcal{C}$ are in $\mathcal{X}$, we can simplify the above to:
\begin{equation}
	c^* = \mbox{argmax}_c \widehat{\mbox{Var}}[(f_{1}(c),..,f_{N}(c))].
	\label{eq:maxvar}
\end{equation}
This procedure combined with BART-BQ is summarised in Algorithm~\ref{alg:SQ}.

\begin{algorithm}[tbh!]
  \caption{Sequential Design}
  \label{alg:SQ}
\begin{algorithmic}
  \STATE {\bfseries Input:}\\training set $\mathcal{D} = \{x_1, \ldots, x_n\}$, \\response $\mathcal{Y} = \{y_1, \ldots, y_n\}$, \\candidate set $\mathcal{C} = \{c_1, \ldots, c_L\},$ 
  \\
  number of iterations $M$
  \FOR{$M$ iterations}
  \STATE fit BART of $K$ trees with $\mathcal{D}$ and $\mathcal{Y}$
  \STATE find $f_{\mbox{BART}}^{(1)}(c), \ldots, f_{\mbox{BART}}^{(K)}(c)$ for all $c\in\mathcal{C}$
  \STATE compute the empirical variance $ \widehat{\mbox{Var}}[(f_{1}(c),..,f_{N}(c))]$
  \STATE Find $\,c^* = \mbox{argmax}_{c} \widehat{\mbox{Var}}[(f_{1}(c),..,f_{N}(c))]
	\label{eq:maxvar}$
  \STATE find response $y_c^* = f(c^*)$
  \STATE $\mathcal{D} \leftarrow \mathcal{D}\cup \{c^*\}$, $\mathcal{C} \leftarrow \mathcal{C}\setminus \{c^*\}$
  \STATE $\mathcal{Y} \leftarrow \mathcal{Y}\cup \{y_c^*\}$
  \ENDFOR
\end{algorithmic}
\end{algorithm}


