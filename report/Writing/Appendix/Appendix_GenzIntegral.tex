\subsection*{Analytical Formulae of the Genz Integrals}
In our experiment, we tested the BQ with BART algorithm on six Genz integrand families \cite{Genz}, whose integral over $[0, 1]^{d}$ can be readily obtained analytically. Here, we provide the analytical formulae for two Genz families whose derivations require a little more effort than the others, namely the Corner Peak and the Oscillatory integrands. Derivations for the rest are straightforward. Note however that the Gaussian peak family is essentially a Gaussian kernel and does not have an analytically tractable integral; one can use the \texttt{pmvnorm} function in the R package \texttt{mvtnorm} to obtain the numerical values.

\subsubsection*{Oscillatory Integrand Family}
For a given dimension $d \in \pmb{N}$, we are interested in deriving the $d$-dimensional integral 
\begin{align}
    I_d = \int_{\pmb{x} \in [0, 1]^{d}} f(\pmb{x}) d\pmb{x} , 
\end{align}
where $u$, $a \in \mathbb{R}$ are function parameters, $\pmb{x} = (x_1, \ ..., \ x_d)^{T}$ and 
\begin{align}
    f(\mathbf{x}) = \cos(2\pi u + a\sum_{i = 1}^{d}x_{i}).
\end{align}

Using mathematical induction (by considering all cases) one can show that, for $d \in \mathbb{N}$, 
\begin{align}
    I_{d} = \frac{1}{a^{d}} \sum_{i = 0}^{d} {d \choose i} (-1)^{d - i} \phi_d(2\pi u + na) ,
\end{align}
where 
\begin{align}
    \phi_d(\cdot) 
    =
    \begin{cases}
    \sin(\cdot) , & d \equiv 1 \bmod 4 ,\\
    -\cos(\cdot) , & d \equiv 2 \bmod 4 ,\\
    -\sin(\cdot) , & d \equiv 3 \bmod 4 ,\\
    \cos(\cdot) , & d \equiv 0 \bmod 4 .
    \end{cases}
\end{align}

\subsubsection*{Corner Peak Integrand Family}
For the Corner Peak family, we proceed as before with the function
\begin{align}
    f(\mathbf{x}) = \left( 1 + a\sum_{i=1}^{d}x_{i} \right)^{-(d + 1)} ,
\end{align}

where $a \in \mathbb{R}$ is a parameter. Again using induction, we obtained the following expression 
\begin{align}
    I_d = \frac{1}{a^{d}d!} \sum_{i = 1}^{d} {d - 1 \choose i - 1}\frac{(-1)^{i - 1}}{(5i + 1)(5i - 4)} .
\end{align}

% \begin{align}
%     f(\mathbf{x}) = (1 + \sum\limits_{i = 1}^{d}a_{i}x_{i})^{- (d + 1)}.
% \end{align}


