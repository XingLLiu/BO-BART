To test our algorithm, we use a set of functions proposed by Genz \cite{Genz}. This is a short list of six families of functions defined via some parameters and used as integrands in testing multidimensional integration algorithms. 

We examine the rate of convergence of our model when estimating the integrals Eq.~\eqref{eq:integral} for each Genz family and compare it to the other two competitive methods, namely Monte Carlo integration and Bayesian quadrature with Gaussian Processes. For simplicity, the distribution $p(x)$ is set to be uniform on $(0, 1)$ (hence, the domain of integration for a $d$-dimensional Genz function is $(0, 1)^{d}$).  We further choose the parameters of the Genz families to be $u = 0.5$ (where applicable) and re-scale $a$ suitably as the dimension increases. Specifically, this is done by bounding the $L_1$-norm of $a$ so that it meets the required difficulty \cite{rescallingGenz}. The resulting mean values and standard deviations for dimensions $1$, $2$, $3$, $5$, $10$ and $20$ are shown in {\color{red} Figure ???}. \\

\subsection{Procedure}

\subsection{Experimental Results}

% We are going to examine our model on how efficient it is in converging to the real value; how reliable the model is in terms of standard deviation and how well the model performs on high dimensional integral.